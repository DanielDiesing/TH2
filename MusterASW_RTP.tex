\documentclass[12pt, oneside, a4paper]{scrreprt}
\usepackage[manualmark,headsepline,footsepline]{scrlayer-scrpage}

\usepackage[german]{babel}
\usepackage[utf8]{inputenc}
\usepackage{graphicx} %Grafik aktivieren
\usepackage{float}
\usepackage[final]{pdfpages}
\usepackage{hyperref}
\usepackage{pdfcomment}
\usepackage{amsmath}
\usepackage{amssymb}
\usepackage{amsfonts}
\newcommand{\dt}{\ensuremath{\mathrm{d}t}}

\usepackage{siunitx}
\usepackage{pgfplots}
\usepackage{pgfplotstable} 
\usepackage{tikz} 
\sisetup{locale = DE, separate-uncertainty}  

\usepackage[backend=biber]{biblatex}
\addbibresource{xmpl.bib}


%% used for pgfplots, tikzpicture
\usepackage{pgfplots}
\pgfplotsset{compat=newest}

%% the following commands are needed for some matlab2tikz features
\usetikzlibrary{plotmarks}
\usetikzlibrary{arrows.meta}
\usepgfplotslibrary{patchplots}
\usepackage{grffile}
\usepackage{amsmath}

\pgfplotsset{plot coordinates/math parser=false}
\newlength\fwidth
  
%%%%%%%%%%%%%%%%%%%%%%%%%%%%%%%%%%%%%%%%%%%%%%%%

%Kapitel-Layout
\usepackage{lmodern} 
\usepackage{xcolor}
\colorlet{chapter}{black!75}
\addtokomafont{chapter}{\color{chapter}}
\makeatletter% siehe De-TeX-FAQ
\renewcommand*{\chapterformat}{%
  \begingroup% damit \unitlength-Änderung lokal bleibt 
  \setlength{\unitlength}{1mm}% 
  \begin{picture}(20,20)(0,5)% 
  \setlength{\fboxsep}{0pt}% 
  \put(20,20){\line(1,0){\dimexpr 
  \textwidth-20\unitlength\relax\@gobble}}% 
  \put(0,0){\makebox(15,20)[r]{% 
  \fontsize{18\unitlength}{28\unitlength}\selectfont\thechapter 
  \kern-.04em% Ziffer in der Zeichenzelle nach rechts schieben 
  }}% 
  \end{picture} % <-- Leerzeichen ist hier beabsichtigt! 
  \newline
  \endgroup
}

\setlength\parindent{0pt}

%%%%%%%%%%%%%%%%%%%%%%%%%%%%%%%%%%%%%%%%%%%%%%%%

% Kopf- und Fußzeilen, Seitenränder etc.
% Kopf- und Fußzeilen ------------------------------------------------------
\pagestyle{scrheadings}

% Kopf- und Fußzeile auch auf Kapitelanfangsseiten -------------------------
\renewcommand*{\chapterpagestyle}{scrheadings}

% Schriftform der Kopfzeile ------------------------------------------------
\renewcommand{\headfont}{%
\normalfont
}

% Kopfzeile ----------------------------------------------------------------
\ihead{Regelungstechnisches Praktikum (RTP)}
\chead{}
\ohead{\includegraphics[scale=1]{Bilder/oth-regensburg-logo.jpg}}
%
\setlength{\headheight}{21mm} % Höhe der Kopfzeile
\setheadsepline[text]{0.4pt} % Trennlinie unter Kopfzeile

% Fußzeile -----------------------------------------------------------------
\ifoot{Jonas Seidl/Daniel Diesing}
\cfoot{\today}
\ofoot{\pagemark}


% erzeugt ein wenig mehr Platz hinter einem Punkt --------------------------
\frenchspacing 

\clubpenalty = 10000
\widowpenalty = 10000 
\displaywidowpenalty = 10000


%%%%%%%%%%%%%%%%%%%%%%%%%%%%%%%%%%%%%%%%%%%%%%%%

\begin{document}

 \begin{titlepage}

	\begin{tabular}{l r} 
 
    \includegraphics[width=0.4\textwidth]{Bilder/oth-regensburg-logo}
      &
		
   \end{tabular}


   \begin{minipage} [c] [8.5cm] [b] {\textwidth}
      \Huge{
         \begin{center}
            \textbf{Praktikum Regelungstechnik (RTP)}\\[0.8cm]
             \par
              {\large Versuch 2 - Systemidentifikation }\\[1cm]
              {\large Praktikumsauswertung} \\[1cm]
          \end{center}
           }
      
      
   \end{minipage}

 \vspace{0.1cm}












	
   \begin{minipage} [c] [3.5cm] [b] {\textwidth}
    \begin{center}
    
    
       \vspace{0.3cm}
       Gruppe 21\\
       Jonas Seidl Matnr.: 3274404 \\
       Daniel Diesing Matnr.: 3278345\\
       \vspace{0.5cm}
       \texttt{Dokument-Version: V1}\\
	\end{center}
  
   \end{minipage}


\end{titlepage}


\tableofcontents

\defineavatar{Newton}{color=lime,subject={Tip},icon=Check,author={J. Seidl /D. Diesing}}
\pdfcommentsetup{avatar=Newton}

%%%%%%%%%%%%%%%%%%%%%%%%%%%%%%%%%%%%%%%%%%%%%%%%%%%%%%%%%%%%%%%%%%%%%

\chapter{Praktikums Vorbereitung}
\label{Praktikumsvorbereitung}
\section{bekannte grundlegende Systemverhalten}
Zur Bestimmung eines Systemverhaltens einer Regelstrecke sowie eines Reglers sind verschiedene Analysemethoden sowie mathematische Verfahren üblich.\\
Die grundlegenden Systemverhalten werden wie folgt benannt und durch verschiedene Glieder wiedergegeben:
\begin{center}
          -Integrierglied\\
          \label{1}
          -Differenzierglied\\
          \label{2}
          -Totzeitglied\\
          \label{3}
          -Proportionalglied\\ 
          \label{4}
          \end{center}
Diese Glieder werden auch elementare Übertragungsglieder genannt, mit der Eigenschaft miteinander verknüpft zu werden. 
\section{Benötigte Parameter zum beschreiben der Systeme}
\begin{itemize}

\item Integrierglied
\begin{equation}
x_a(t)=K_I\int x_e(t)\dt
\end{equation}
$K_I=\text{Integralbeiwert}$
\item Differenzierglied
\begin{equation}
x_a(t)=K_D\frac{\mathrm{d}}{\dt} x_e(t)
\end{equation}
$K_D=\text{Diffenrentialbeiwert}$
\item Totzeitglied
\begin{equation}
x_a=K_tx_e(t-T_t)
\end{equation}
$K_t=\text{Totzeitbeiwert}$
\item Proportionalglied
\begin{equation}
x_a(t)=K_P\cdot x_e(t)
\end{equation}
$K_P=\text{Proportionalbeiwert}$

\end{itemize}
\section{Beschreibung zweier Verhalten der elementaren Übertragungsglieder}

\chapter{Versuchsauswertung}
\section{System 1}
In der zur Verfügung stehenden Messreihe wird ein System beschrieben. Im kommenden wird dies zuerst mit MatLab ausgewertet und geplottet, im Anschluss mit Simulink rekonstruiert. 
\subsection{Auswertung per MatLab}





\begin{figure}[H]
\centering
\pgfplotstableread[col sep = comma]{Sys1_TEX.txt}{\Systemverhalten}
\pgfplotstableread[col sep = comma]{simulation.txt}{\Simulation}
\begin{tikzpicture}
\begin{axis}[
 	title = \textbf{Systemverhalten $PT_1T$},
	xmin = 0, xmax = 9,
    ymin = -3.5, ymax = 0.5,
    xtick distance = 1,
    ytick distance = 0.5,
    grid = both,
    minor tick num = 1,
    major grid style = {lightgray},
    minor grid style = {lightgray!25},
    width = 0.75\textwidth,
    height = 0.6\textwidth,
    xlabel = {$t/\si{\second}$},
    ylabel = {$x_a(t)$}],
    legend pos = noth east,
    legend cell align = {left}
\addplot[
    thin,
    blue
 ]table[x=time, y=behaviour] {\Systemverhalten};%\addlegendentry{Systemverhalten}
 \addplot[
  thin,
 red
 ]table[x=time, y=behav] {\Simulation};%\addlegendentry{Simulation}
 \legend{Systemverhalten, Simulation}
\end{axis}
\end{tikzpicture}
\caption{Vergleich eines realen und simulierten Systemverhalten}
\end{figure}

Diese Darstellung eaines Signals weist einige Eigenschaften. Einerseits ist zu sehen, dass in den ersten \SI{1.5}{\second} eine Totzeit zu sehen ist. Des Weiteren wird deutlich, dass anschließend die Funktion fällt. Zusätzlich ist zu erkennen, dass das Signal mit einem Rauschen versehen ist.
Ein neuer Satz.


\end{document}
